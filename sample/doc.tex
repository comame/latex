\documentclass[11pt, uplatex]{jsarticle}
\usepackage{mystyle}

\EquationNumberWithinSection

\begin{document}

\begin{abstract}
    ここに摘要
\end{abstract}
\hidePageNumber
\clearpage

% 目次を挿入
\tableofcontents
\hidePageNumber
\clearpage

\Title{Document Title}
\Center{Center}
\Right{Right}

\section{Section}
\subsection{Subsection}
\subsubsection{Subsubsection}

Lorem ipsum dolor sit amet, consectetur adipiscing elit, sed do eiusmod tempor incididunt ut labore et dolore magna aliqua. Ut enim ad minim veniam, quis nostrud exercitation ullamco laboris nisi ut aliquip ex ea commodo consequat. Duis aute irure dolor in reprehenderit in voluptate velit esse cillum dolore eu fugiat nulla pariatur. Excepteur sint occaecat cupidatat non proident, sunt in culpa qui officia deserunt mollit anim id est laborum.

表\ref{table-label}を参照。図\ref{image-label}を参照。式\eqref{expr}を参照。インライン数式 $ 1 + 1 = 2 $。

\begin{equation}
    \label{expr}
    x^2 + 2x + 4 = 0
\end{equation}

\begin{displaymath}
    x^2 + 2x + 4 = 0
\end{displaymath}

\begin{eqnarray}
    x ^ 2 + 4x + 4 &=& 0 \nonumber \\
    (x + 2) ^ 2 &=& 0 \nonumber \\
    x &=& -2
\end{eqnarray}

\begin{eqnarray*}
    x ^ 2 + 4x + 4 &=& 0 \\
    (x + 2) ^ 2 &=& 0 \\
    x &=& -2
\end{eqnarray*}

\begin{equation}
    \U{斜体キャンセル}
\end{equation}

\begin{equation}
    1.5 \E{6}
\end{equation}

\begin{Table}{*{3}c}{table-name}{table-label}
    \hline \hline
    Left & Center & Right \\
    \hline
    1 & 2 & 3 \\
    1 & 2 & 3 \\
    \multicolumn{2}{c}{multi} & 3 \\
    \hline
\end{Table}

\begin{Table*}{*{3}{c}}
    \hline \hline
    Left & Center & Right \\
    \hline
    1 & 2 & 3 \\
    1 & 2 & 3 \\
    \multicolumn{2}{c}{multi} & 3 \\
    \hline
\end{Table*}

\begin{equation}
    x^2 + 2x + 4 = 0
\end{equation}

\begin{equation}
    x^2 + 2x + 4 = 0
\end{equation}

\Figure{img/trp.png}{image-name}{image-label}{10cm}
\Figure*{img/trp.png}{10cm}

\addtocounter{table}{-1} % これでカウンターをデクリメント

インラインコード \Code{console.log('Hello, world!')}。

\begin{lstlisting}[caption=ソースコード, label=sourcecode]
    const str = 'Hello, world'
    console.log(str)
\end{lstlisting}

ソースコード \ref{sourcecode} を参照。

文献を参照する \cite{refLabel}。

\begin{Refs}
    \RefItem{refLabel} 文献
    \RefItem{refLabel2} URL はこんな感じ \url{https://example.com}
\end{Refs}

\end{document}
